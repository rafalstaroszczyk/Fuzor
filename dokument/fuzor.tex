\documentclass[onecolumn]{article}
\usepackage[utf8]{inputenc}
\usepackage[T1]{fontenc}
\usepackage{indentfirst}
\usepackage{float}
\usepackage{amsmath}  % równania
\usepackage{amssymb}
\usepackage{bbold}
\usepackage{physics2}  % pochodne, macierze itp
\usephysicsmodule{ab}
\usephysicsmodule{diagmat}
\usephysicsmodule{xmat}
\usephysicsmodule{nabla.legacy}
\usephysicsmodule{op.legacy}
\makeatletter
\newcommand\vb{\@ifstar\boldsymbol\mathbf}
\newcommand\va[1]{\@ifstar{\vec{#1}}{\vec{\mathrm{#1}}}}
\newcommand\vu[1]{%
\@ifstar{\hat{\boldsymbol{#1}}}{\hat{\mathbf{#1}}}}
\makeatother
\usepackage{fixdif, derivative}  % pochodne

\newcommand{\inv}[1]{\frac{1}{#1}}  % odwrotność

\begin{document}
\section{Podstawy elektrodynamiczne}
\begin{center}
Równania Maxwella
\end{center}
\begin{align*}
\div\vb{E} &= \inv{\varepsilon_{0}}\rho & \curl\vb{E} &= -\pdv{\vb{B}}{t} \\
\div\vb{B} &= 0 & \curl\vb{B} &= \mu_{0}\vb{j} + \mu_{0}\varepsilon_{0}\pdv{\vb{E}}{t}
\end{align*}
Z równania \(\div\vb{B} = 0\) możemy wprowadzić potencjał wektorowy \(\vb{A}\), taki że:
\begin{equation*}
\vb{B} = \curl\vb{A}.
\end{equation*}
Po podstawieniu tego do równania \(\curl\vb{E} = -\pdv{\vb{B}}{t}\) i manipulacji otrzymujemy 
\begin{equation*}
\vb{E} = -\grad{V} - \pdv{\vb{A}}{t},
\end{equation*}
gdzie \(V\) jest potencjałem skalarnym. \par
Podstawiając te potencjały do wzorów \(\div\vb{E} = \inv{\varepsilon_{0}}\rho\) oraz \(\curl\vb{B} = \mu_{0}\vb{j} + \mu_{0}\varepsilon_{0}\pdv{\vb{E}}{t}\):
\begin{gather*}
\laplacian V + \pdv{}{t}\pab{\div\vb{A}} = -\inv{\varepsilon_{0}}\rho \\
\laplacian \vb{A} - \grad\pab{\div\vb{A}} - \mu_{0}\varepsilon_{0}\div\pdv{V}{t} - \mu_{0}\varepsilon_{0}\pdv[order=2]{\vb{A}}{t} = -\mu_{0}\vb{j}.
\end{gather*}
Zapisując powyższe wzory w bardziej symetrycznej formie otrzymujemy:
\begin{gather*}
\pab{\laplacian V - \mu_{0}\varepsilon_{0}\pdv[order=2]{V}{t}} + \pdv{}{t}\pab{\div\vb{A} + \mu_{0}\varepsilon_{0}\pdv{V}{t}} = -\inv{\varepsilon_{0}}\rho \\
\pab{\laplacian \vb{A} - \mu_{0}\varepsilon_{0}\pdv[order=2]{\vb{A}}{t}} - \div\pab{\div\vb{A} + \mu_{0}\varepsilon_{0}\pdv{V}{t}} = -\mu_{0}\vb{j}.
\end{gather*}
Wybierając cechowanie Lorentza
\begin{equation*}
\div\vb{A} + \mu_{0}\varepsilon_{0}\pdv{V}{t} = 0,
\end{equation*}
mamy:
\begin{gather*}
\laplacian V - \mu_{0}\varepsilon_{0}\pdv[order=2]{V}{t} = -\inv{\varepsilon_{0}}\rho \\
\laplacian \vb{A} - \mu_{0}\varepsilon_{0}\pdv[order=2]{\vb{A}}{t} = -\mu_{0}\vb{j}.
\end{gather*}
\section{Układ symetryczny cylindrycznie}
%\subsection{Cząstka w polu elektrostatycznym}
%\subsection{Wiele cząstek bez wzajemnych oddziaływań}
%\subsection{Wiele cząstek ze wzajemnymi oddziaływaniami}

\section{Układ symetryczny sferycznie}
\subsection{Cząstka w polu elektrostatycznym}

%\subsection{Wiele cząstek bez wzajemnych oddziaływań}
%\subsection{Wiele cząstek ze wzajemnymi oddziaływaniami}
\end{document}